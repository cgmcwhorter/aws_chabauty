% !TEX root = ../../../main/aws_chabauty.tex
\newpage
\subsection{Lecture 2}
\subsubsection{Selmer Groups}

	\[
	0 \ma{} J[p] \ma{} J \ma{p} J \ma{} 0
	\]
	\[
	J(K) \ma{p} J(K) \ma{} H^1(K, J[p])
	\]
	\[
	\dfrac{J(K)}{pJ(K)} \hookrightarrow H^1(K,J[p])
	\]
But the right side is an infinite dimensional $\F_p$-vector space if $\dim J \geq 0$. 

	\[
	\begin{tikzcd}
	\dfrac{J(K)}{pJ(K)} \arrow[hook]{r} & \sel_p J \arrow[draw=none]{r}[sloped,auto=false]{\subseteq} & H^1(K,J[p]) \arrow{d}{\beta} \\
	\prod_\nu \dfrac{J(K_\nu)}{pJ(K_\nu)} \arrow[hook]{rr}{\alpha} & & \prod_\nu H^1(K_\nu,J[p])
	\end{tikzcd}
	\]
$\sel_p J:= \{ \xi \in H^1(K,J[p]) \colon \beta(\xi) \in \im \alpha \}$. This group is conjecturally finite and computable. 


Similarly,
	\[
	\dfrac{J(K)}{p^nJ(K)} \hookrightarrow \sel_{p^n} J \subset H^1(K,J[p^n])
	\]
By taking inverse limits
	\[
	\hat{J(K)} \hookrightarrow \sel_{\Z_p} J \subset H^1(K,T)
	\]
then by inverting $p$
	\[
	\hat{J(K)}\left[\frac{1}{p}\right] \hookrightarrow \sel_{\Q_p} J \subset H^1(K,V)
	\]
Then we have
	\[
	0 \ma{} \dfrac{J(K)}{pJ(K)} \ma{} \sel_p J \ma{} \sha[p] \ma{} 0
	\]
	
	\[
	0 \ma{} \hat{J(K)}\left[\frac{1}{p}\right] \ma{} \sel_{\Qp} J \ma{} \left( \plim \sha[p^n] \right)\left[\frac{1}{p}\right] \ma{} 0
	\] 


	\[
	\begin{tikzcd}
	X(K) \arrow{d} \arrow{r} & X(K_\p) \arrow{d} \\
	J(K) \arrow{r} \arrow{d} & J(K_\p) \arrow{r}{\log} \arrow{d} & \lie J_{K_\p} \\
	\widehat{J(K)}[\frac{1}{p}] \arrow{r} \arrow{d} & \widehat{J(K_\p)}[\frac{1}{p}] \arrow{ur}{\rotatebox{45}{$\sim$}} \\
	\sel_{\Q_p} J \arrow{d} \\
	H^1(K,V) \arrow{r} & H^1(K_\p,V)
	\end{tikzcd}
	\]



% Bloch-Kato Selmer Groups
\subsubsection{Bloch-Kato Selmer Group}

We examine the Block-Kato Selmer group in terms of $V$ and not $J$. In the general setting (local Galois representations), let $V$ be a finite dimensional $\Q_p$-vector space with continuous action---$G_{K_\nu}$-action. 
	\[
	D_{\cris}(V):= (B_{\cris} \otimes_{\Q_p} V)^{G_{K_\nu}}
	\]
where $B_{\cris}$ is a certain ring equipped with a $G_{K_\nu}$-action. 


\begin{rem}
$\dim_{K_\nu} D_{\cris}(V) \leq \dim_{\Q_p} V$
\end{rem}


\begin{dfn}[Crystalline]
We call $V$ crystalline if equality holds. 
\end{dfn}


\begin{rem}
For $\nu$ and an abelian variety $J/K_\nu$, then $J$ has good reduction if and only if its $\Q_p$ Tate module $V$ is unramified if $\nu \nmid p$ and crystalline if $\nu \mid p$. 
\end{rem}


Now suppose $\xi \in H^1(K,V)$. Let
	\[
	0 \ma{} V \ma{} E \ma{} \Q_p \ma{} 0
	\]
be the corresponding extension. Call $\xi$ crystalline if $E$ is crystalline. 
	\[
	H_f^1(K_\nu,V):= \{ \text{crystalline classes in } H^1(K_\nu,V) \}
	\]


\begin{rem}
$\p \mid p$. If $J$ is an abelian variety with good reduction at $\p$, and $V$ is a $\Q_p$ Tate module, then the iamge of
	\[
	\hat{J(K_\p)}[\frac{1}{p}] \to H^1(K_\p,V)
	\]
equals $H_f^1(K_\p,V)$. If $\p \nmid p$, then $H^1(K_\p,V= 0$.
\end{rem}



% Global Galois Representations
\subsubsection{Global Galois Representations}

Let $V$ be a finite dimensional $\Q_p$-vector space with continuous $G_K$-action. Given $\xi \in H^1(K,V)$. Let $\xi_\nu$ be its image in $H^1(K_\ni,V)$. The Bloch-Kato Selmer group of $v$ is the group
	\[
	H_f^1(K,V):= \{ \xi \in H^1(K,V) \colon \xi_\nu \text{ is crystalline for all } \nu \mid p \}
	\]


\begin{rem}
If $J$ is an abelian variety over $K$ and $V$ is a $\Q_p$ Tate module, then $H_f^1(K,V)= \sel_{\Q_p} J$. 
\end{rem}


	\[
	\begin{tikzcd}
	X(K) \arrow{d} \arrow{r} & X(K_\p) \arrow{d} \\
	J(K) \arrow{r} \arrow{d} & J(K_\p) \arrow{r}{\log} \arrow{d} & \lie J_{K_\p} \\
	\widehat{J(K)}[\frac{1}{p}] \arrow{r} \arrow{d} & \widehat{J(K_\p)}[\frac{1}{p}] \arrow{ur}{\rotatebox{45}{$\sim$}} \arrow{d} \\
	\sel_{\Q_p} J \arrow{d}= H_f^1(K,V) \arrow{r} &  H_f^1(K_\p,V) \arrow{d} \\
	H^1(K,V) \arrow{r} & H^1(K_\p,V)
	\end{tikzcd}
	\]


Algebraic de Rham Cohomology: $H_\dr^1(X):= \H^1(X,\Omega^0)$ with Hodge filtration $\Fil^0$, where $\H$ is hypercohomology. Define also $H_1^\dr(X):=$ dual of $H_\dr^1$ with dual filtration. 


	\[
	\begin{tikzcd}
	X(K) \arrow{d} \arrow{r} & X(K_\p) \arrow{d} \\
	J(K) \arrow{r} \arrow{d} & J(K_\p) \arrow{r}{\log} \arrow{d} & \lie J_{K_\p} \\
	\widehat{J(K)}[\frac{1}{p}] \arrow{r} \arrow{d} & \widehat{J(K_\p)}[\frac{1}{p}] \arrow{ur}{\rotatebox{45}{$\sim$}} \arrow{d} \\
	\sel_{\Q_p} J \arrow{d}= H_f^1(K,V) \arrow{r} &  H_f^1(K_\p,V) \arrow{d} \arrow{r}{\log, \sim} &  H_1^\dr(K_{K_\p})/\Fil^0 \\
	H^1(K,V) \arrow{r} & H^1(K_\p,V)
	\end{tikzcd}
	\]


We get 
	\[
	\begin{tikzcd}
	X(K) \arrow{r} \arrow{d} & X(K_\p) \arrow{d} \arrow{dr}{p\text{-adic integrals}} \\
	H_f^1(K,V) \arrow{r} &  H_f^1(K_\p,V) \arrow{r}{\sim} & H_1^\dr(X_{K_\p})/\Fil^0
	\end{tikzcd}
	\]



% Lower Central Series
\subsubsection{Lower Central Series}

Let $G$ be a (topological) group. For $A, B \leq g$, define $(A,B):= \ov{\langle aba^{-1}b^{-1} \colon a \in A, b \in B \rangle}$. Now define the lower central series by
	\[
	\begin{aligned}
	C^1G&:= G \\
	C^2G&:= (G,C^1G)= (G,G) \\
	C^3G&:= (G,C^2G) \\
	&\phantom{=}\vdots
	\end{aligned}
	\]
Finally, define $G_n:= G/C^{n+1}G$. This is a $n$-step nilpotent group. 


\begin{ex}
$G_1= G/(G,G)=: G^{\ab}$, the abelianization of $G$, i.e. the largest abelian quotient. 
\end{ex}


This was just Group Theory. Now let's apply this to the fundamental group of the curve.



% Abelianized Fundamental Group
\subsubsection{Abelianized Fundamental Group}

Given $M$ a connected real manifold, $m \in M$, we get $\pi_1(M,m)^\ab \simeq H_1(M,\Z)$, where $\pi_1$ is the fundamental group. What is the algebraic version? Given $X$, a `nice' curve of genus $g$ curve over $K$, $x \in X(K)$, we obtain
	\[
	\pi_1^\et(X_{\ov{K}}, x)^\ab \simeq H_1^\et(X_{\ov{K}}, \hat{\Z})
	\]
But we have maps
	\[
	\pi_1^\et(X_{\ov{K}},x)_1 \ma{\sim} \pi_1^\et(X_{\ov{K}}, x)^\ab \simeq H_1^\et(X_{\ov{K}}, \hat{\Z}) \sra H_1^\et(X_{\ov{K}}, \Z_p) \subset H_1^\et(X_{\ov{K}}, \Q_p)=: V= \cV(\Q_p)
	\]


Kim obtains a generalization
	\[
	\pi_1^\et(X_{\ov{K}},x) \ma{} V_n= \cV_n(\Q_p)
	\]
where $\cV_n$ is some unipotent algebraic group, and
	\[
	\begin{tikzcd}
	X(K) \arrow{d} \arrow{r} & X(K_\p) \arrow{d} \arrow{dr}{p\text{-adic iterated integrals}} \\
	H_f^1(K,V_n) \arrow{r} & H_f^1(K_\p,V_n) \arrow{r}{\sim} & \pi_1^\dr(X_{K_\p},x)_n/\Fil^0
	\end{tikzcd}
	\]
and morphisms of $\Q_p$-varieties
	\[
	\sel^{[n]} \ma{} J^{[n]} \ma{} L^{[n]}
	\]
which gives you the $\Q_p$ points of $\pi_1^\dr(X_{K_\p},x)_n/\Fil^0$. 


\begin{thm}[Kim]
If for some $n \geq 1$, $\dim \sel^{[n]} < \dim J^{[n]}$, then $X(K)$ is contained in the set of zeros of some nonzero locally analytic functions on the local points of the curve, which are given by some iterated integrals. Therefore, $X(K)$ is finite.
\end{thm}















