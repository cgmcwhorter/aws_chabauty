% !TEX root = ../../../main/aws_chabauty.tex
\newpage
\subsection{Lecture 2}
\subsubsection{Selmer Groups}

Last time, we had the following diagram which we wanted to rephrase in terms of something without Jacobians.
	\[
	\begin{tikzcd}
	X(K) \arrow{d} \arrow{r} & X(K_\p) \arrow{d} \\
	J(K) \arrow{r} \arrow{d} & J(K_\p) \arrow{r}{\log} \arrow{d} & \lie J_{K_\p} \\
	\widehat{J(K)}[\frac{1}{p}] \arrow{r} & \widehat{J(K_\p)}[\frac{1}{p}] \arrow{ur}{\rotatebox{45}{$\sim$}} 
	\end{tikzcd}
	\]
To begin, we defined a $\Qp$ Tate module, which we called $V$, defined in terms of the \etale homology of the curve. Now we want to relate the rational points to these Galois cohomology groups using Selmer groups and the descent map. 


We begin with the Kummer sequence for this Jacobian
	\[
	0 \ma{} J[p] \ma{} J \ma{p} J \ma{} 0
	\]
Now taking Galois cohomology yields
	\[
	J(K) \ma{p} J(K) \ma{} H^1(K, J[p])
	\]
We can contract this as
	\[
	\dfrac{J(K)}{pJ(K)} \hookrightarrow H^1(K,J[p])
	\]
Originally, this was done to prove the Mordell-Weil theorem by bounding $J(K)/pJ(K)$, which was the first step in proving that $J(K)$ is finitely generated. It would be nice if $H^1(K,J[p])$ were finite; however, $H^1(K,J[p])$ is an infinite dimensional $\F_p$-vector space if $\dim J \geq 0$. But perhaps, studying the image of this inclusion more carefully, it can be shown that there is some finite subspace of $H^1(K,J[p])$ which bounds $J(K)/pJ(K)$, or at least figure out which of the classes in $H^1(K,J[p])$ come from $J(K)/pJ(K)$. The idea will be not to look at what is in the image directly but rather look what is in the image locally. So using functoriality, we can write down the following commutative diagram  
	\[
	\begin{tikzcd}
	\dfrac{J(K)}{pJ(K)} \arrow[hook]{r} \arrow{d} & H^1(K,J[p]) \arrow{d} \\
	\dfrac{J(K_\nu)}{pJ(K_\nu)} \arrow[hook]{r} & H^1(K_\nu,J[p])
	\end{tikzcd}
	\]
For a class in $H^1(K,J[p])$ to have come from $J(K)/pJ(K)$, it must appear in the inclusion into $H^1(K_\nu,J[p])$ by following the commutativity of the diagram. So we have a condition for classes from $J(K)/pJ(K)$ to appear in $H^1(K,J[p])$. Of course, we can do this for all $\nu$, in which case we obtain the following diagram, which will give us the Selmer group.
	\[
	\begin{tikzcd}
	\dfrac{J(K)}{pJ(K)} \arrow[hook]{r} \arrow{d} & H^1(K,J[p]) \arrow{d} \\
	\prod_\nu \dfrac{J(K_\nu)}{pJ(K_\nu)} \arrow[hook]{r} & \prod_\nu H^1(K_\nu,J[p])
	\end{tikzcd}
	\]
The $p$-Selmer group, which is a subgroup of the global Galois cohomology $H^1(K,J[p])$ which contains the image of the inclusion $J(K)/pJ(K) \hra H^1(K,J[p])$.
	\[
	\begin{tikzcd}
	\dfrac{J(K)}{pJ(K)} \arrow[hook]{r} \arrow{d} & \sel_p J \arrow[draw=none]{r}[sloped,auto=false]{\subseteq} & H^1(K,J[p]) \arrow{d}{\beta} \\
	\prod_\nu \dfrac{J(K_\nu)}{pJ(K_\nu)} \arrow[hook]{rr}{\alpha} & & \prod_\nu H^1(K_\nu,J[p])
	\end{tikzcd}
	\]


\begin{dfn}[$p$-Selmer group]
The $p$-Selmer group, denoted $\sel_p J$, is $\{ \xi \in H^1(K,J[p]) \colon \beta(\xi) \in \im \alpha \}$. 
\end{dfn}


\begin{rem}
This group is finite and computable. Moreover, this gives the only currently known way of bounding ranks of abelian varieties over number fields. Other arguments reduce to this method in some way. 
\end{rem}


We can repeat this same argument using powers of $p$.
	\[
	\dfrac{J(K)}{p^nJ(K)} \hookrightarrow \sel_{p^n} J \subset H^1(K,J[p^n])
	\]
By taking inverse limits, we obtain the $p$-adic completion of the Mordell-Weil group.\footnote{In Iwasawa Theory, direct limits are taken instead of inverse limits.}
	\[
	\hat{J(K)} \hookrightarrow \sel_{\Z_p} J \subset H^1(K,T)
	\]
where $T$ is the Tate module. Again to avoid rings which are not fields, we invert $p$ to obtain
	\[
	\hat{J(K)}\left[\frac{1}{p}\right] \hookrightarrow \sel_{\Qp} J \subset H^1(K,V)
	\]
The Selmer groups are acting as upper bounds for the Jacobians. But we want to say more about the `error' in the approximation. The obstruction, or error, here is the Shafarevich-Tate group, $\sha$. Classically, we have the group which we are trying to compute, $J(K)$, the `approximation' $\sel_p J$, and the `error' $\sha[p]$.
	\[
	0 \ma{} \dfrac{J(K)}{pJ(K)} \ma{} \sel_p J \ma{} \sha[p] \ma{} 0
	\]
Proceeding just as we did above, i.e. taking inverse limits and inverting $p$, we obtain
	\[
	0 \ma{} \hat{J(K)}\left[\frac{1}{p}\right] \ma{} \sel_{\Qp} J \ma{} \left( \plim \sha[p^n] \right)\left[\frac{1}{p}\right] \ma{} 0
	\] 
The group $\sha$ is conjecturally finite. But then, conjecturally, the groups $\sha[p^n]$ stabilize for sufficiently large $n$. The maps in $\left( \plim \sha[p^n] \right)\left[\frac{1}{p}\right]$ are multiplication by $p$. So the term $\left( \plim \sha[p^n] \right)\left[\frac{1}{p}\right]$ is conjecturally zero if $\sha[p^n]$ is finite. But then $\hat{J(K)}\left[\frac{1}{p}\right]$ is a $\Qp$-vector space of dimension equal to the rank, exactly approximated by $\sel_{\Qp} J$. 


Notice we still have not succeeded in completely eliminating Jacobians. However, we have made some progress in our original diagram by injecting Selmer groups. 
	\[
	\begin{tikzcd}
	X(K) \arrow{d} \arrow{r} & X(K_\p) \arrow{d} \\
	J(K) \arrow{r} \arrow{d} & J(K_\p) \arrow{r}{\log} \arrow{d} & \lie J_{K_\p} \\
	\widehat{J(K)}[\frac{1}{p}] \arrow{r} \arrow{d} & \widehat{J(K_\p)}[\frac{1}{p}] \arrow{ur}{\rotatebox{45}{$\sim$}} \\
	\sel_{\Qp} J \arrow{d} \\
	H^1(K,V) \arrow{r} & H^1(K_\p,V)
	\end{tikzcd}
	\]
The next step will be to find a way of describing the Selmer group $\sel_{\Qp} J$ that does not involve $J$.



% Bloch-Kato Selmer Groups
\subsubsection{Bloch-Kato Selmer Group}

We want to describe $\sel_{\Qp}$ solely in terms of $V$ and not $J$. Rather than immediately discuss the $V$ we had talked about before, we will speak in the more general setting of local Galois representations. Let $V$ be a finite dimensional $\Qp$-vector space with continuous $G_{K_\nu}$-action, where $G_{K_\nu}:= \Gal(\overline{K}_\nu/K_\nu)$. We first define an important object created by Fontaine
	\[
	D_{\cris}(V):= (B_{\cris} \otimes_{\Qp} V)^{G_{K_\nu}}
	\]
where $B_{\cris}$ is a certain ring equipped with a $G_{K_\nu}$-action. We will not worry ourselves precisely what this is, which requires a discuss (at the very least) of Witt vectors. Now it is a fact that $\dim_{K_\nu} D_{\cris}(V) \leq \dim_{\Qp} V$. We say that $V$ is crystalline if, in fact, equality holds. 


To get an idea of what this condition is, fix a $\nu$ and consider an abelian variety $J/K_\nu$, then $J$ has good reduction if and only if its $\Qp$ Tate module $V$ is unramified (if $\nu \nmid p$) or crystalline (if $\nu \mid p$). We will need one other definition, so suppose $\xi \in H^1(K,V)$. The cohomology class is naturally associated to an extension of $\Qp$ by $V$, namely\footnote{Note that $E$ here has nothing to do with elliptic curves\dots it is just the first letter of the word extension.}
	\[
	0 \ma{} V \ma{} E \ma{} \Qp \ma{} 0
	\]
What is the correspondence? These are all local Galois representations. Now $\Qp$ has a trivial action. Given $\xi$, we can write down a vector space $E$ with the extensions given as above. Going the other way, given the extension as above, we take the element of $H^0(K_\nu,\Qp)$ from $\Qp$, namely 1. This element will map to $H^1(K_\nu,V)$, which will be our $\xi$. In any case, let the sequence above be the corresponding extension. We call $\xi$ crystalline if $E$ is crystalline (because $E$ is a local Galois representation). We could define $\xi$ to be crystalline if it is in the kernel of the map from $H^1(K_\nu,V)$ to $H^1(K_\nu, B_\cris \otimes V)$. We now can define a subset of all the crystalline cohomology classes, as Block-Kato did. So define $H_f^1(K_\nu,V):= \{ \text{crystalline classes in } H^1(K_\nu,V) \}$. 


\begin{rem}
Suppose that $\p \mid p$. If $J$ is an abelian variety with good reduction at $\p$, and $V$ is a $\Qp$ Tate module, then the image of the Kummer descent map $\hat{J(K_\p)}[\frac{1}{p}] \to H^1(K_\p,V)$ is $H_f^1(K_\p,V)$. [If $\p \nmid p$, then $H^1(K_\p,V)= 0$.] This is a generalization of the fact that for elliptic curves, if you look at a prime of good reduction and you look at the image of the local  points inside the local Galois cohomology, you obtain exactly the unramified classes. In our case here, we obtain the crystalline classes. 
\end{rem}



% Global Galois Representations
\subsubsection{Global Galois Representations}

We now have a way of talking about local points without referring to the Jacobian anymore. Namely, the image of the local points of the Jacobian can be characterized just in terms of $V$ by using crystalline cohomology classes. We need to do this globally, rather than just locally. Let $V$ be a finite dimensional $\Qp$-vector space with continuous $G_K$-action. Given a global cohomology class $\xi \in H^1(K,V)$, we can obtain a local cohomology class $\xi_\nu$, namely the image of $\xi$ in $H^1(K_\nu,V)$. We define the Bloch-Kato Selmer group of $v$ to be the group $H_f^1(K,V):= \{ \xi \in H^1(K,V) \colon \xi_\nu \text{ is crystalline for all } \nu \mid p \}$, i.e. the set of global cohomology classes satisfying the local conditions. 


We can now re-express the Selmer group in Bloch-Kato terms, without referring to $J$. If $J$ is an abelian variety over $K$ and $V$ is a $\Qp$ Tate module, then the Bloch-Kato Selmer group is $H_f^1(K,V)= \sel_{\Qp} J$, which is just the classical $\Qp$ Selmer group of $J$. Now we can talk about the Selmer group $\sel_{\Qp} J$ using $H_f^1(K,V)$, which depends only on $X$ because $V$ depends only on $X$. We can now return to our diagram and fill in even more by replacing $\sel_{\Qp} J$ with $H_f^1(K,V)$, mapping $\widehat{J(K_\p)}[\frac{1}{p}]$ to $H_f^1(K_\p,V)$, which are subgroups of $H^1(K,V)$ and $H^1(K_\p,V)$, respectively. We have the restriction map $H_f^1(K,V) \to H_f^1(K_\p,V)$ and local coboundary map $\widehat{J(K_\p)}[\frac{1}{p}] \to H_f^1(K_\p,V)$. This gives us the following diagram
	\[
	\begin{tikzcd}
	X(K) \arrow{d} \arrow{r} & X(K_\p) \arrow{d} \\
	J(K) \arrow{r} \arrow{d} & J(K_\p) \arrow{r}{\log} \arrow{d} & \lie J_{K_\p} \\
	\widehat{J(K)}[\frac{1}{p}] \arrow{r} \arrow{d} & \widehat{J(K_\p)}[\frac{1}{p}] \arrow{ur}{\rotatebox{45}{$\sim$}} \arrow{d} \\
	\sel_{\Qp} J \arrow{d}= H_f^1(K,V) \arrow{r} &  H_f^1(K_\p,V) \arrow{d} \\
	H^1(K,V) \arrow{r} & H^1(K_\p,V)
	\end{tikzcd}
	\]














% 31:46

The only thing left to do is remove the tangent space at the identity, $\lie J_{K_\p}$. For this, we will need one more definition. 


Algebraic de Rham Cohomology: $H_\dr^1(X):= \H^1(X,\Omega^0)$ with Hodge filtration $\Fil^0$, where $\H$ is hypercohomology. Define also $H_1^\dr(X):=$ dual of $H_\dr^1$ with dual filtration. 


	\[
	\begin{tikzcd}
	X(K) \arrow{d} \arrow{r} & X(K_\p) \arrow{d} \\
	J(K) \arrow{r} \arrow{d} & J(K_\p) \arrow{r}{\log} \arrow{d} & \lie J_{K_\p} \\
	\widehat{J(K)}[\frac{1}{p}] \arrow{r} \arrow{d} & \widehat{J(K_\p)}[\frac{1}{p}] \arrow{ur}{\rotatebox{45}{$\sim$}} \arrow{d} \\
	\sel_{\Qp} J \arrow{d}= H_f^1(K,V) \arrow{r} &  H_f^1(K_\p,V) \arrow{d} \arrow{r}{\log, \sim} &  H_1^\dr(K_{K_\p})/\Fil^0 \\
	H^1(K,V) \arrow{r} & H^1(K_\p,V)
	\end{tikzcd}
	\]


We get 
	\[
	\begin{tikzcd}
	X(K) \arrow{r} \arrow{d} & X(K_\p) \arrow{d} \arrow{dr}{p\text{-adic integrals}} \\
	H_f^1(K,V) \arrow{r} &  H_f^1(K_\p,V) \arrow{r}{\sim} & H_1^\dr(X_{K_\p})/\Fil^0
	\end{tikzcd}
	\]



% Lower Central Series
\subsubsection{Lower Central Series}

Let $G$ be a (topological) group. For $A, B \leq g$, define $(A,B):= \ov{\langle aba^{-1}b^{-1} \colon a \in A, b \in B \rangle}$. Now define the lower central series by
	\[
	\begin{aligned}
	C^1G&:= G \\
	C^2G&:= (G,C^1G)= (G,G) \\
	C^3G&:= (G,C^2G) \\
	&\phantom{=}\vdots
	\end{aligned}
	\]
Finally, define $G_n:= G/C^{n+1}G$. This is a $n$-step nilpotent group. 


\begin{ex}
$G_1= G/(G,G)=: G^{\ab}$, the abelianization of $G$, i.e. the largest abelian quotient. 
\end{ex}


This was just Group Theory. Now let's apply this to the fundamental group of the curve.



% Abelianized Fundamental Group
\subsubsection{Abelianized Fundamental Group}

Given $M$ a connected real manifold, $m \in M$, we get $\pi_1(M,m)^\ab \simeq H_1(M,\Z)$, where $\pi_1$ is the fundamental group. What is the algebraic version? Given $X$, a `nice' curve of genus $g$ curve over $K$, $x \in X(K)$, we obtain
	\[
	\pi_1^\et(X_{\ov{K}}, x)^\ab \simeq H_1^\et(X_{\ov{K}}, \hat{\Z})
	\]
But we have maps
	\[
	\pi_1^\et(X_{\ov{K}},x)_1 \ma{\sim} \pi_1^\et(X_{\ov{K}}, x)^\ab \simeq H_1^\et(X_{\ov{K}}, \hat{\Z}) \sra H_1^\et(X_{\ov{K}}, \Z_p) \subset H_1^\et(X_{\ov{K}}, \Qp)=: V= \cV(\Qp)
	\]


Kim obtains a generalization
	\[
	\pi_1^\et(X_{\ov{K}},x) \ma{} V_n= \cV_n(\Qp)
	\]
where $\cV_n$ is some unipotent algebraic group, and
	\[
	\begin{tikzcd}
	X(K) \arrow{d} \arrow{r} & X(K_\p) \arrow{d} \arrow{dr}{p\text{-adic iterated integrals}} \\
	H_f^1(K,V_n) \arrow{r} & H_f^1(K_\p,V_n) \arrow{r}{\sim} & \pi_1^\dr(X_{K_\p},x)_n/\Fil^0
	\end{tikzcd}
	\]
and morphisms of $\Qp$-varieties
	\[
	\sel^{[n]} \ma{} J^{[n]} \ma{} L^{[n]}
	\]
which gives you the $\Qp$ points of $\pi_1^\dr(X_{K_\p},x)_n/\Fil^0$. 


\begin{thm}[Kim]
If for some $n \geq 1$, $\dim \sel^{[n]} < \dim J^{[n]}$, then $X(K)$ is contained in the set of zeros of some nonzero locally analytic functions on the local points of the curve, which are given by some iterated integrals. Therefore, $X(K)$ is finite.
\end{thm}















