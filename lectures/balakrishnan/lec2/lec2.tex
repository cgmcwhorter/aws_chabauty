% !TEX root = ../../../main/aws_chabauty.tex
\newpage
\subsection{Lecture 2}

Last time, we gave an algorithm to compute Coleman integrals between points in different residue disks on hyperelliptic curves, i.e. the  curve's ``analytic continuation along Frobenius.'' To do this, we wrote down an action of Frobenius $\omega$ on differentials $\omega_i$, reduced pole orders, and obtained $\phi^* w_i= dh_i + \sum M_{ji} \omega_j$, then wrote a system to produce 
	\[
	\left\{
	\begin{pmatrix}
	\vdots \\
	\ds \int_Q^P \omega_i \\
	\vdots
	\end{pmatrix} 
	\right\}_{i=0,\ldots,2g-1}
	\]


How do we do this for more general curves? We use Tuitman's algorithm, showing how it can be used to compute Coleman integrals for plane curves. Let $X/\Q$ be a `nice' curve of genus $g$ with a plane model $Q(x,y)= y^{d_x} + Q_{d_x-1} y^{d_x-1} + \cdots + Q_0= 0$ such that $Q(x,y)$ is irreducible, $Q_i(x) \in \Z[x]$. Let $p$ be a good prime for $X$.


\noindent\makebox[\textwidth][c]{%
\begin{minipage}{0.90\textwidth}
\begin{enumerate}[1.]
\item Consider the map $x: X \to \P^1$, and remove the ramification locus $r(x)$ of $x$---analogous to removing Weierstrass points in Kedlaya's algorithm. 
\item Choose a lift of Frobenius with $x \mapsto x^p$. Compute the image of $y$ through Hensel lifting. 
\item Compute a basis of $H_{\dr}^1(X)$ using an integral basis of $\Q(X)$ over $\Q[X]$, $\Q[\frac{1}{x}]$.
\item Compute the action of Frobenius on differentials and reduce pole orders using relations in cohomology via Louder's fibration algorithm---Tuitman uses an integral basis of $\Q(X)$. 
\end{enumerate}
\end{minipage}%
} \par\vspace{\baselineskip}


Then we had $\phi^*\omega_i= dh_i + \sum M_{ji} \omega_j$. We then used this to give a linear system to produce values
	\[
	\begin{pmatrix}
	\vdots \\
	\ds\int_Q^P \omega_i \\
	\vdots
	\end{pmatrix}
	\]


\begin{ex}[Balakrishnan-Tuitman]
Can compute Coleman integrals on a non-hyperelliptic curve with genus 55 to show its Jacobian has rank $\geq 1$. \xqed
\end{ex}


Let $X/\Q$ be a `nice' curve of genus $g$. By work of Coleman (1982) and Coleman-de Shalit (1988), we have a theory of iterated $p$-adic integrals on $X$. These are iterated path integrals:
	\[
	\int_P^Q \eta_n \cdots \eta_1:= \int_0^1 \int_0^{t_1} \cdots \int_0^{t_{n-1}} f_n(t_n) \cdots f_1(t_1) \;dt_n \; \cdots \; dt_1
	\]
We will often suppress writing the iterated integral and instead write a single integral. In our computations, we will focus on the case $n=2$ (double Coleman integrals):
	\[
	\int_P^Q \eta_2 \eta_1:= \int_P^Q \eta_2(R) \int_P^R \eta_1
	\]
These integrals play an important role in nonabelian Chabauty.


How do we compute these integrals? We want to apply an algorithm for computing the action of Frobenius on $p$-adic cohomology, e.g. Kedlaya's algorithm or Tuitman's algorithm, to produces
	\[
	\phi^* \omega_i= dh_i + \sum_{j=0}^{2g-1} M_{ji} \omega_j
	\]
Observe that the algorithm of $M^{\otimes n}$ are not 1, and reduce the computation of an $n$-fold iterated integrals to a computation of an $(n-1)$-fold iterated integrals. Here are some useful properties of iterated Coleman integrals:


\begin{prop}
Let $\omega_{i_1}, \ldots, \omega_{i_n}$ be forms of the second kind, holomorphic at $P,Q \in X(\Qp)$. Then

\begin{enumerate}[(i)]
\item $\ds\int_P^P \omega_{i_1} \cdots \omega_{i_n}= 0$
\item
	\[
	\sum_{\text{all perm.}} \int_P^Q \omega_{\sigma(i_1)} \cdots \omega_{\sigma(i_n)}= \prod_{j=1}^n \int_P^Q \omega_{i_j}
	\]
\item $\ds\int_P^Q \omega_{i_1} \cdots \omega_{i_n}= (-1)^n \int_Q^P \omega_{i_n} \cdots \omega_{i_1}$
\item If $P,P', Q \in X(\Qp)$, then
	\[
	\int_P^Q \omega_{i_1} \cdots \omega_{i_n}= \sum_{j=0}^n \int_{P'}^Q \omega_{i_1} \cdots \omega_{i_j} \int_P^{P'} \omega_{i_{j+1}} \cdots \omega_{i_n}
	\]
\end{enumerate}
\end{prop}


% iv This let's us break up a path.?

So this gives the analogue of additivity in endpoints for double integrals, i.e. if $P,P',Q \in X(\Qp)$), then
	\[
	\int_P^Q \omega_i \omega_k= \int_P^{P'} \omega_i \omega_k + \int_{P'}^{Q'} \omega_i \omega_k + \int_{Q'}^Q \omega_i \omega_k + \int_P^{P'} \omega_k \int_{P'}^Q \omega_i + \int_{P'}^{Q'} \omega_k + \int_{Q'}^Q \omega_i
	\]
We can compute double Coleman integrals directly using a linear system. Let $P'= \phi(P)$ and $Q'= \phi(Q)$. Then
	\[
	\begin{aligned}
	\int_{\phi(P)}^{\phi(Q)} \omega_i \omega_k&= \int_P^Q \phi^*(\omega_i \omega_k) \\
	&=\int_P^Q \phi^*(\omega_i) \phi^*(\omega_k) \\
	&= \int_P^Q (df_i + \sum_{j=0}^{2g-1} M_{ji} \omega_j )( \;df_k  + \sum M_{jk} \omega_j ) \\
	&= c_{ik} + \int_P^Q \left(\sum_{j=0}^{2g-1} M_{ji} \omega_j\right) \left(\sum_{j=0}^{2g-1} M_{jk} \omega_j \right)
	\end{aligned}
	\]
where $c_{ik}$ depends only on single integrals, see the notes. This gives us
	\[
	\begin{pmatrix}
	\vdots \\
	\ds \int_P^Q \omega_i \omega_k \\
	\vdots
	\end{pmatrix}=
	(I_{4g^2} - (M^t)^{\otimes 2})^{-1}
	\begin{pmatrix}
	\vdots \\
	\ds c_{ik} - \int_{\phi(P)}^{P} \omega_i \omega_k - \int_P^Q \omega_i \int_{\phi(P)}^P \omega_i - \int_Q^{\phi(Q)} \omega_i - \int_{\phi(P)}^{\phi(Q)} \omega_k + \int_{\phi(Q)}^Q \omega_i \omega_k \\
	\vdots
	\end{pmatrix}
	\]


As an application (or perhaps a preview), let $E/\Z$ be the minimal regular model of an elliptic curve. Let $\cX= \cE \setminus \O$. Let $\omega_0= \dfrac{dx}{2y+a_1x+a_3}$, $\omega_1= x\omega_0$ in Weierstrass coordinates. Let $b$ be a tangential base point of $\O$ at an integral 2-torsion point, and let $p$ be a prime of good reduction. Suppose that $\cE$ has analytic rank 1 and Tamagawa product 1. Let
	\[
	\begin{aligned}
	\log(z)&= \int_b^z \omega_0 \\
	D_2(z)&= \int_b^z \omega_0 \omega_1
	\end{aligned}
	\]
Think of $\log z$ as the Coleman integral extending $\log$ on the formal group of $\cE/\Z_p$. The notational choice for $D_2$ is to remind one of a dilogarithm. 


\begin{thm}[Kim,Balakrishnan-Kedlaya-Kim]
Suppose $P$ is a point of infinite order in $\cE(\Z)$. Then $\cX(\Z) \subseteq \cE(\Z)$ is in the zero set of
	\[
	f(z)= (\log P)^2 D_2(z) - (\log z)^2 D_2(P)
	\]
i.e., $\dfrac{D_2(z)}{(\log z)^2}$ is constant on integral points. 
\end{thm}


Kim shows that this $D_2$ is related to $p$-adic heights on $\cE$, which is what we will discuss next. 



% $p$-adic heights on Jacobians of curves
\subsubsection{$p$-adic heights on Jacobians of curves}

From a computational viewpoint, quadratic Chabauty seeks to replace the linear relations that make it possible to cut out rational points among $p$-adic points in Chabauty-Coleman by bilinear relations. However, $p$-adic heights are a natural source of bilinear forms on global points; these heights allow us to generate some of our linear techniques from Chabauty-Coleman.


\begin{rem}
Some $p$-adic heights are in $p$-adic BSD/$p$-adic Gross-Zagier.
\end{rem}


Let $X/\Q$ be a `nice' curve of genus $g>1$, and let $p$ be a prime of good reduction. Fix a branch of $\log_p: \Qp^\times \to \Qp$. Fix also an \idele class character $\chi: A_\Q^*/\Q^* \to \Qp$ and a splitting $s$ of the Hodge filtration on $H_\dr^1(X/\Qp)$ such that $\ker(s)$ is isotropic with respect to the cup product. 


\begin{rem}
Fixing a splitting of the Hodge filtration corresponds to fixing a subspace $W:= \ker s$ of $H^1_\dr(X)$ complementary to the space of holomorphic forms $H^0(X_{\Qp},\Omega^1)$, i.e. $H^1_\dr(X/\Qp) \simeq H^0(X_{\Qp},\Omega^1) \oplus W$
\end{rem}


\begin{dfn}[Coleman-Gross, 1989]
The cyclotomic $p$-adic height pairing is a symmetric biadditive pairing
	\[
	\begin{aligned}
	\div^0(X) \times \div^0(X) &\to \Qp \\
	(D_1,D_2) &\mapsto h(D_1,D_2)
	\end{aligned}
	\]
for $D_1,D_2 \in \div^0(X)$ with disjoint support such that

\begin{enumerate}[(i)]
\item 
	\[
	h(D_1,D_2)= \sum_{\text{finite primes }v} h_v (D_1,D_2)= h_p(D_1,D_2) + \sum_{\ell \neq p} h_\ell (D_1,D_2)= \int_{D_2} \omega_{D_1} + \sum m_\ell \log_p (\ell)
	\]
where the integral is a Coleman integral, the sum is finite, and $m_\ell \in \Q$ is an intersection multiplicity. 
\item For $\beta \in \Q(X)^*$, we have $h(D,\ldiv(\beta))=0$.
\end{enumerate}
\end{dfn}


By the definition (specifically (ii)), this gives a symmetric, bilinear pairing $J(\Q) \times J(Q) \to \Qp$.
 


% Local heights at $P$
\subsubsection{Local heights at $P$}

We will say a bit more about the local height pairings $h_v$ in the case where $v= p$. First, we need to construct the normalized differential $\omega_D$ with respect to choice of $\omega$ in (i) above. Let $T(\Qp)$ be the group of differentials of the third kind on $X$: simple poles and integer residues. We have a residue divisor homomorphism
	\[
	\begin{aligned}
	\res: T(\Qp) &\to \div^0(X) \\
	\omega &\mapsto \res(\omega) = \sum_P (\res_P \omega) P
	\end{aligned}
	\]
This induces a short exact sequence
	\[
	0 \ma{} H^0(X_{\Qp}, \Omega^1) \ma{} T(\Qp) \ma{\res} \div^0(X) \ma{} 0
	\]
We want $\omega_{D_1}$ to be a certain third kind of differential with $\res(\omega_{D_1})= D_1$. It turns out that this is indeed the case. 


\begin{ex}
Let $X$ be a hyperelliptic curve $y^2= f(x)$, and $D_1$ is the divisor given by $D_1= (P) - (Q)$, where $P,Q$ are non-Weierstrass points. We want a simple pole at $P,Q$ with residues $+1, -1$, respectively. Now
	\[
	\omega= \dfrac{dx}{2y} \left( \dfrac{y+y(P)}{x-x(P)} - \dfrac{y+y(Q)}{x-x(Q)} \right)
	\]
has $\res \div D_1$  a simple pole at $P,Q$ residues $+1,-1$, respectively. There are no other poles. However, adding any holomorphic differential $\eta$ to $\omega$, we still have $\res(\eta+\omega)= D_1$. We must take care of this by fixing a particular 1-form. We want $\omega$ to be complementary to the space of 1-forms and also be isotropic with respect to the cup product pairing. \xqed
\end{ex}


How will we normalize? Let $T_\ell(\Qp)$ be the group of logarithmic differentials $\dfrac{df}{P}$ with $f \in \Qp(X)^*$. We know that $T_\ell(\Qp) \cap H^0(X_{\Qp},\Omega^1)= 0$ and $\res \frac{df}{f}= \div f$. Then combining this with the short exact sequence from above, we have 
	\[
	0 \ma{} H^0(X_{\Qp}, \Omega^1) \ma{} T(\Qp)/T_l(\Q_p) \ma{} J(\Q_p) \ma{} 0
	\] 


\begin{prop}
There is a canonical homomorphism $\Psi: T(\Q_p)/T_l(\Qp) \to H_\dr^1(X)$ such that
        \begin{enumerate}[(i)]
        \item $\Psi$ is the identity on holomorphic differentials, i.e. differentials of the first kind.
        \item $\Psi$ sends third kind differentials to second kind differentials, modulo exact differentials.
        \end{enumerate}
\end{prop}


\begin{dfn}[$\omega_D$]
Let $D \in \div^0(X)$. We define $\omega_D$ to be the unique differential of the third kind with  $\res(v_0)= D$ and $\psi(\omega_0) \in W$.
\end{dfn}


\begin{rem}
If $p$ a prime of ordinary reduction for the Jacobian, we can take $W$ to be the unit root subspace for Frobenius. In fact, if $\phi$ is a lift of Frobenius, then $\{ (\phi^*)^n \omega_g, \ldots, (\phi^*)^n\omega_{2g-1}\}$ is a basis for the unit root subspace modulo $p^n$.
\end{rem}


What we will need to do is integrate differentials of the third kind to give an algorithm for computing local $p$-adic heights. 

